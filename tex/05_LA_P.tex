\documentclass{article}
\usepackage{amsfonts, amsmath, amssymb}
\usepackage{xparse}

\author{  }
\date{ 2021-03-13 }
\title{ Linearna Algebra 5 }

\begin{document}
\maketitle
    \section*{ Vektorski prostor: def }
    (Realni) vektrosski prostor \(V\) je množica elementov, ki jih imienujemo vektorji \(v \in V\), skupaj z dvema notranjima operacijama: 
    - seštevanje vektorjev 

    - množenje vektorjev s skalarjem 
    z lastnostmi operacij med vektorji

    \section*{ Vektorski prostor: lastnostni }

    - linearna kombinacija 
    - \(v \in V, \ 0 \in \mathbb{R} \Rightarrow 0 \cdot v = 0\) 

    - \(0 \in V, \ \alpha \in \mathbb{R} \Rightarrow \alpha \cdot 0 = 0 \) 
    - \(\alpha \cdot v = 0 \Rightarrow \alpha = 0 \ ali \ v = 0 \)

    \section*{ Vektorski prostor: linearna kombinacija }
    :: \(u,v \in V \ \alpha, \beta \in \mathbb{R} \)


    Za \(v_1, ..., v_n \in V\) imenujemo vsak izraz \(\alpha_1 v_1 + \alpha_2 v_2 + ... \alpha_n v_n\) 

    linearna kombbbinacija vektorjev \(v_1, ..., v_n\). 
    Vsaka linearna kombinacija iz \(V\) je tudii v \(V\).

    \section*{ Vektorski podprostor: definicija }
    :: \(V\) vektorski prostor 
    Podmnožica \(u \subseteq V\) je vektorki podprostor v \(V\), če: 
    - \(+\) je notranja operacija: \(u_1, u_2 \in U \Rightarrow u_1 + u_2 \in U\) 
    - \(\cdot\) je notranja operacija: \(u \in U, \ \alpha \in \mathbb{R} \Rightarrow \alpha u \in U\)

    \section*{ Vektorski podprostor: lastnosti }
    - vsakk vektorski prostor vsebuje \(0\) 
    - vsak vektor v vektorskem podprostoru vsebuje nasprotni elementi v tem podprostoru 
    - vssak vektorski podprostor je tudi vektorski prostor

    \section*{ Vektorski podprostor: trditev }
    \(U\) podnožica vektroskega prostora \(V\): 
    - \(U\) zaprt za seštevanje in množenje s skalarjem 
    - \(U\) zaprt za linerne kombinacije

    \section*{ Simetrična matrika: def }
    Za matrike \(\mathbb{R}^{n \times n}\) pravimo, da je simetrična, če velja, je enaka \(A = A^T\) 
    \((A=[a_{ij}]: a_{ij} = a_{ji})\)

    \section*{ Ničelni prostor matrike: def }
    Za matriko \(A \in \mathbb{R}^{m \times n}\) imenujemo 
    \(N(A) = \{\vec{x} \in \mathbb{R}^n; A\vec{x} = \vec{0}\} \subseteq \mathbb{R}^n\) ničelni prostor matrike \(A\) 
    - \(N(1)\) je vektorski podprostor v \(\mathbb{R}^n\)

    \section*{ Linearna ogrinjača: def }
    :: \(V\) vektorski prostor in \(v_1, ..., v_k \in V\) 
    \(L\{ v_1, ..., v_u  \} = \{ \alpha_1 v_1 + ... + \alpha_v; \alpha_1, ..., \alpha_k \in \mathbb{R}\}\) 
    imenujemo linearna ogrinjača 
    
    \section*{ Linearna ogrinjača: lastnosti }
    - je najmanjši vektroski podprostor v \(V\), ki vsebuje vektorje \(v_1, ..., v_k\) 
    - vektorji \(v_1, ..., v_k\) napenjajo / razpenjajo linearno ogrinjačo

    \section*{  }

    \section*{  }

    \section*{  }

    \section*{  }

    \section*{  }

    \section*{  }

    \section*{  }

    \section*{  }

    \section*{  }

    \section*{  }

    \section*{  }

    \section*{  }

    \section*{  }

    \section*{  }

    \section*{  }

    \section*{  }

    \section*{  }

    \section*{  }

    \section*{  }

    \section*{  }

\end{document}